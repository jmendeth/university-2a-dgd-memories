\documentclass[catalan,parskip=half*,oneside,DIV=11,hidelinks]{scrreprt}

% encoding
\usepackage[utf8]{inputenc}
\usepackage[T1]{fontenc}
\usepackage{lmodern}
\usepackage{babel}

% formatting and fixes
\frenchspacing
\usepackage[style=spanish]{csquotes}
\MakeAutoQuote{«}{»}
\usepackage{bookmark}
\usepackage{scrhack}

% ADD ANY SPECIFIC PACKAGES HERE
% (CHEMISTRY, CODE, PUBLISHING)
\usepackage{environ}
\usepackage{enumitem}
\usepackage{xcolor}
\usepackage{tcolorbox}
\usepackage{adjustbox}
\usepackage{booktabs}
\usepackage[figure]{hypcap}

\usepackage{siunitx}
\usepackage{mathtools}
\usepackage{circuitikz}
\usetikzlibrary{calc}
\usetikzlibrary{automata}
\usetikzlibrary{circuits.ee.IEC}
\usetikzlibrary{circuits.logic.US,circuits.logic.IEC}
\usetikzlibrary{arrows.meta}
\usepackage{tikz-timing}
%\usepackage{karnaugh}
\usepackage{minted}
\usepackage{../SevenSeg}

% hyperlink setup / metadata
\usepackage{hyperref}
\AfterPreamble{\hypersetup{
  pdfauthor={Àlex Moreno; Xavier Méndez},
  pdftitle={Memòria pràctica 3},
  pdfsubject={DGD},
  linkcolor={red!50!black},
  citecolor={blue!50!black},
  urlcolor={blue!80!black},
  bookmarksopen=true,
}}
\setlength{\footskip}{2.0cm}

% custom commands: BDF schematic and BSF symbol
\newcommand{\nodenamebit}[1]{\text{#1}}
\newcommand{\nodenamesingle}[2]{\text{#1}_{#2}}
\newcommand{\nodenamerange}[3]{\text{#1}_{#2..#3}}

\newcommand{\bdfschematic}[1]{
\begin{tikzpicture}[
  circuit ee IEC, circuit logic US,
  input pin/.style={},
  output pin/.style={},
  symbol/.style={},
  primitive/.style={symbol},
  symbol type/.style={font=\sffamily\scriptsize, gray},
  node line/.style={>={Latex[length=3pt, width=3pt]}},
  bus line/.style={>={Latex[length=4pt, width=5pt]}, line width=1.4pt},
  line name/.style={font=\small, inner sep=0pt},
  pin name/.style={font=\small, inner sep=7pt},
  port name/.style={font=\scriptsize, inner sep=0pt},
  junction/.style={},
  pin bounds/.style={symbol bounds},
  symbol bounds/.style={draw=gray, dotted},
]
\input{../\projectname/assets/schematics/#1.tex}
\end{tikzpicture}
}

\newcommand{\bsfsymbol}[1]{
\begin{tikzpicture}[
  circuit ee IEC, circuit logic US, scale=1.3,
  input pin/.style={},
  output pin/.style={},
  symbol/.style={},
  primitive/.style={symbol},
  symbol type/.style={font=\sffamily\small, gray},
  node line/.style={>={Latex[length=3pt, width=3pt]}},
  bus line/.style={>={Latex[length=4pt, width=5pt]}, line width=1.4pt},
  line name/.style={inner sep=0pt},
  pin name/.style={inner sep=7pt},
  port name/.style={font=\small, inner sep=0pt},
  junction/.style={},
  pin bounds/.style={symbol bounds},
  symbol bounds/.style={draw=none},
]
\input{../\projectname/assets/symbols/#1.tex}
\end{tikzpicture}
}

% custom commands: VHD listing

\definecolor{bg}{rgb}{0.95,0.95,0.95}
\newcommand{\vhdlisting}[1]{
\inputminted[
  tabsize=4, breaklines=true, framesep=10pt, frame=leftline, breakindent=2em,
]{vhdl}{../\projectname/assets/vhdl/#1.vhd}
}

% custom commands: whole block Tex file

\newcommand{\inputblock}[1]{
\input{../\projectname/blocks/#1.tex}
}

% custom commands: port list
\SetLabelAlign{myright}{\hss\llap{$#1$}}
\newlist{where}{description}{1}
\setlist[where]{labelwidth=3cm,labelsep=1em,
                        leftmargin=!,align=myright,font=\normalfont}


% custom commands: controlled figure placement
\gdef\accumcontent{}
\newcommand{\accumfig}[1]{%
  \begin{figure}[p]#1\end{figure}%
}
\newcommand{\dumpaccum}{\accumcontent \gdef\accumcontent{}}
\NewEnviron{contendfig}{%
  \expandafter\expandafter\expandafter\def\expandafter\expandafter\expandafter\accu \expandafter\expandafter\expandafter{\expandafter\accumcontent\expandafter\accumfig\expandafter{\BODY}} %
  \global\let\accumcontent\accu %
}

% custom commands: inline seven-segments
\newcommand{\sevenseg}[1]{%
\marginbox{.1em .2em}{\begin{tikzpicture}[
  baseline={([yshift=-.5ex]current bounding box.center)},
  SSGOff/.style={gray!20!white, line width=2pt},
  SSGOn/.style={SSGOff, black},
]%
\SSGNb[.7em]{}{#1}%
\end{tikzpicture}}%
}

% add PDF bookmark for index
\let\myTOC\tableofcontents
\renewcommand\tableofcontents{%
  \pdfbookmark[1]{\contentsname}{}
  \myTOC }



\begin{document}

\title{Memòria pràctica 3}
\date{11 de desembre de 2016}
\author{Àlex Moreno; Xavier Méndez}

\maketitle

\tableofcontents
\clearpage


\newcommand{\cclearpage}{}
\newcommand{\projectname}{}
\newcommand{\inputproject}[1]{
  \dumpaccum
  \renewcommand{\projectname}{#1}
  \chapter{Pràctica 3: Disseny d'un joc}

\section{Especificació}

Es demana implementar un petit joc en què l'objectiu del jugador és endevinar un
nombre aleatori que la màquina manté ocult. Per fer-ho, introdueix un nombre i el
sistema comprova si coincideix amb la solució. Si no coincideix, li comunica si
aquest nombre és major o menor que aquesta. El jugador va fent aproximacions
successives fins que encerta.

En iniciar-se el sistema, aquest es troba en un estat de repós, amb tots els LEDs
verds encesos i un comptador intern incrementant el seu valor de forma cíclica.
En el moment que l'usuari prem la tecla \texttt{\#}, el comptador s'atura
(determinant així la solució a encertar), els LEDs s'apaguen i comença la partida.

Un cop començada la partida, l'usuari pot teclejar als dígits del teclat el
nombre que desitja provar. Un cop introduïts, pitja \texttt{*} i el sistema
mostra als LEDs verds el resultat de la comparació:

\begin{itemize}
\item Si ha encertat (comparació igual), el sistema passa immediatament a l'estat
   de repós inicial (però sense encendre tots els LEDs, per no esborrar el resultat
   de la comparació). Llavors només és necessari prèmer \texttt{\#} un cop per
   començar una partida nova.

\item Si no ha encertat (comparació major o menor), només s'actualitzen els LEDs
   per mostrar el resultat de la comparació. Els LEDs es tornaran a apagar quan
   l'usuari premi un dígit de nou.
\end{itemize}

Mentre la partida està activa, l'usuari pot premer \texttt{\#} en qualsevol moment;
llavors s'encenen tots els LEDs de nou i es torna a l'estat de repós. Un segon \texttt{\#}
inicia una partida nova.

Els patrons de LEDs son:

\begin{center} \begin{tabular}{cl}
\topline
Patró & Significat \\
\midline
\ledeightpattern{X}{X}{X}{X}{X}{X}{X}{X} & Joc en estat de repós, cap partida iniciada. \\
\ledeightpattern{ }{ }{ }{ }{ }{ }{ }{ } & Partida iniciada, s'està esperant a la comprovació d'un nombre. \\
\ledeightpattern{X}{X}{X}{X}{ }{ }{ }{ } & El nombre introduït és menor a la solució. \\
\ledeightpattern{ }{ }{ }{ }{X}{X}{X}{X} & El nombre introduït és major a la solució; partida acabada. Joc en estat de repós. \\
\ledeightpattern{ }{ }{X}{X}{X}{X}{ }{ } & El nombre introduït és igual a la solució; partida acabada. Joc en estat de repós. \\
\bottomline
\end{tabular} \end{center}

Es farà servir una placa DE2 on es carregarà el resultat. Es farà servir el Quartus 9.1 SP2 Web Edition per al desenvolupament.

\section{Implementació}

% TODO

\section{Blocs importats}

En aquesta secció es llisten els blocs que s'han importat de pràctiques
anteriors i es fan servir sense modificar-los (ni modificar les seves
dependències). Per aquest motiu, no s'inclou la implementació ni la
simulació d'aquests.

Un bloc importat només apareix si es fa servir directament en algun bloc
no importat. Per tant, les seves dependències no apareixen necessàriament
en aquesta memòria.

\inputblock{imported/f_div}
\inputblock{imported/keytest}
\inputblock{imported/BCD7seg}
\inputblock{imported/hex_disps}

\section{Blocs del joc}
\inputblock{joc/keygroup}
\inputblock{joc/comptador}
\inputblock{joc/registres}
\inputblock{joc/comparador}
\inputblock{joc/control}
\inputblock{joc}

\section{Blocs d'adaptació a la placa}
\inputblock{leds}
\inputblock{jocDE2}

\section{Valoració general}

L'únic problema rellevant era que el comptador feia l'overflow en les unitats
però no en les desenes (passava de \texttt{0x99} a \texttt{0xA0}), i de vegades
el joc quedava impossible de resoldre. Solucionat això, tot va funcionar correctament.

}
\inputproject{3}
\inputproject{3-extra1}
\inputproject{3-extra2}
\inputproject{3-extra3}

\end{document}
