\subsection{\label{sub:\projectname-keygroup} \textsf{keygroup}}

\paragraph{Símbol}

\begin{center} \bsfsymbol{keygroup} \end{center}

\paragraph{Entrades i sortides}

\begin{where}
\item[\nodenamebit{nkey}] Senyal que s'activa si s'ha premut una tecla (actiu baix)
\item[\nodenamerange{x}{3}{0}] Índex de la tecla que s'ha premut
\item[\nodenamebit{bcd}] Senyal que s'activa si s'ha premut una xifra decimal
\item[\nodenamebit{ast}] Senyal que s'activa si s'ha premut la tecla asterisc
\item[\nodenamebit{coi}] Senyal que s'activa si s'ha premut la tecla coixinet
\item[\nodenamebit{let}] Senyal que s'activa si s'ha premut una lletra
\end{where}

\paragraph{Funció}

Evalua la tecla premuda $x$, si n'hi ha, i activa una (o cap) de les sortides següents,
segons el tipus de tecla premuda:

\begin{itemize}
\item $bcd$ si és un dígit decimal. En aquest cas, $x$ és també
el valor d'aquest dígit en BCD.
\item $ast$ si és la tecla asterisc (\texttt{*}).
\item $coi$ si és la tecla coixinet (\texttt{\#}).
\item $let$ si és una lletra.
\end{itemize}

\paragraph{Inespecificacions}

Cap.

\paragraph{Implementació}

\vhdlisting{keygroup}



Cada sortida estarà activa només si $nkey$ està actiu i $x$ és el valor adequat.
Per al cas de $bcd$ i $let$ no és un valor únic sino un rang, però es pot escriure de
forma compacta mitjançant una comparació.

\paragraph{Simulació}

\begin{center}
  \begin{tikztimingtable}[timing/rowdist=4ex]
  $\nodenamebit{nkey}$  &  [] 45L \\
  $\nodenamebit{x}$  &  [] 4.5D{1001} 4.5D{1111} 4.5D{0101} 4.5D{1010} 4.5D{0110} 4.5D{1011} 4.5D{1110} 4.5D{1101} 4.5D{0111} 4.5D{1111} \\
  $\nodenamebit{bcd}$  &  [] 4.529106H 4.515345L 4.5H 4.5L 4.5H 13.5L 4.5H 4.45555L \\
  $\nodenamebit{ast}$  &  [] 18.022933L 0.015293H 8.993724L 4.490849H 13.477201L \\
  $\nodenamebit{coi}$  &  [] 4.523008L 4.499866H 18.000134L 0.009003H 8.990863L 0.015427H 4.5L 4.461698H \\
  $\nodenamebit{let}$  &  [] 4.529277L 0.000148H 4.499852L 0.015441H 4.5L 4.484707H 4.515293L 4.484707H 4.499852L 4.5H 8.970723L \\
\extracode
\end{tikztimingtable}

\end{center}

Ens fixem que detecti les lletres activant la nova sortida.

\vspace{1cm}
