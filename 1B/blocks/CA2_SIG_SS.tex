\subsection{\label{sub:\projectname-CA2_SIG_SS} \textsf{CA2\_SIG\_SS}}

\paragraph{Símbol}
\begin{center} \bsfsymbol{CA2_SIG_SS} \end{center}

\paragraph{Entrades i sortides}

\begin{where}
\item[\nodenamebit{sig}] Signe a representar (1 $\rightarrow$ negatiu)
\item[\nodenamerange{ss}{6}{0}] Sortida pel display (set-segments, actiu baix)
\end{where}

\paragraph{Funció}

Representa el signe donat en un patró per a display set-segments actiu baix.

\paragraph{Inespecificacions}

Cap.

\paragraph{Implementació}

\begin{figure}[b]
  \begin{center}
    \adjustbox{max width=\textwidth, max height=\textheight}{
      \bdfschematic{CA2_SIG_SS}
    }
  \end{center}
  \caption{\label{fig:\projectname-CA2_SIG_SS} Esquemàtic per al bloc \textsf{CA2\_SIG\_SS}}
\end{figure}

L'esquemàtic del bloc es pot veure a la figura~\ref{fig:\projectname-CA2_SIG_SS} (pàgina~\pageref{fig:\projectname-CA2_SIG_SS}).

Tots els segments del display estan apagats, excepte el segment del mig, que
s'ilumina quan $sig = 1$. Per tant, com que son actius baixos, la sortida és
$(\overline{sig}, 1, 1, 1, 1, 1, 1)$.

\vspace{1cm}
