\subsection{\label{sub:\projectname-MULT_ACC} \textsf{MULT\_ACC}}

\paragraph{Símbol}
\begin{center} \bsfsymbol{MULT_ACC} \end{center}

\paragraph{Entrades i sortides}

\begin{where}
\item[\nodenamerange{a}{7}{0}] Primer factor
\item[\nodenamebit{b}] Segon factor
\item[\nodenamebit{z}] Bit de sortida
\item[\nodenamerange{in}{7}{0}] Carry d'entrada
\item[\nodenamerange{out}{7}{0}] Carry de sortida
\end{where}

\paragraph{Funció}

Multiplicador $8 \times 1$ amb carry.

Multiplica els primer factor de 8~bits pel segon factor d'un bit, amb carry
d'entrada $in$ i retorna el bit resultant a la sortida $z$ i el carry de
sortida en $out$.

Nota: Aquest és un bloc auxiliar que s'encadena per a fer
multiplicadors $8 \times N$.

\paragraph{Inespecificacions}

Cap.

\paragraph{Implementació}

\begin{figure}[b]
  \begin{center}
    \adjustbox{max width=\textwidth, max height=\textheight}{
      \bdfschematic{MULT_ACC}
    }
  \end{center}
  \caption{\label{fig:\projectname-MULT_ACC} Esquemàtic per al bloc \textsf{MULT\_ACC}}
\end{figure}

L'esquemàtic del bloc es pot veure a la figura~\ref{fig:\projectname-MULT_ACC} (pàgina~\pageref{fig:\projectname-MULT_ACC}).

Al principi es multipliquen els dos factors amb un \textsf{MULT\_8x1}, i el
resultat es suma amb el carry d'entrada. Dels 9~bits de sortida de la suma (carry
inclós) el de menys pes es retorna a la sortida $z$ i els altres 8, al carry de
sortida.

\vspace{1cm}
