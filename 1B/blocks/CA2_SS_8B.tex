\subsection{\label{sub:\projectname-CA2_SS_8B} \textsf{CA2\_SS\_8B}}

\paragraph{Símbol}
\begin{center} \bsfsymbol{CA2_SS_8B} \end{center}

\paragraph{Entrades i sortides}

\begin{where}
\item[\nodenamerange{x}{7}{0}] Entrada (Ca2)
\item[\nodenamerange{hex0}{6}{0}] Primer display en ordre lèxic (set-segments actiu baix, signe)
\item[\nodenamerange{hex1}{6}{0}] Segon display en ordre lèxic (set-segments actiu baix, desenes)
\item[\nodenamerange{hex2}{6}{0}] Tercer display en ordre lèxic (set-segments actiu baix, unitats)
\end{where}

\paragraph{Funció}

Formatejador de Ca2 8~bits en set-segments.

Donat un valor en complement a 2 de 8~bits, el representa en format signe --
mòdul decimal habitual en 3 displays set-segments $hex0$, $hex1$ i $hex2$.

Nota: Els displays estan numerats en ordre lèxic ($hex0$ és el de l'esquerra).

\paragraph{Inespecificacions}


A causa de \textsf{CA2\_BCD\_8B}, les sortides $hex1$ i $hex2$ són indefinides
quan $x$ està fora del rang $\left[-56, 64\right]$.


\paragraph{Implementació}

\begin{figure}[b]
  \begin{center}
    \adjustbox{max width=\textwidth, max height=\textheight}{
      \bdfschematic{CA2_SS_8B}
    }
  \end{center}
  \caption{\label{fig:\projectname-CA2_SS_8B} Esquemàtic per al bloc \textsf{CA2\_SS\_8B}}
\end{figure}

L'esquemàtic del bloc es pot veure a la figura~\ref{fig:\projectname-CA2_SS_8B} (pàgina~\pageref{fig:\projectname-CA2_SS_8B}).

En primer lloc, el signe $x_3$ es fa entrar al bloc \textsf{CA2\_SIG\_SS} i la
representació resultant es retorna a $hex0$. Llavors, s'utilitza el bloc
\textsf{CA2\_BCD\_8B} per a obtenir el mòdul de $x$ en BCD. Cada una de les dues
xifres del mòdul es converteix a set-segments i es retorna a $hex1$ i $hex2$
respectivament.

\vspace{1cm}
