\subsection{\label{sub:\projectname-CA2_BCD_4B} \textsf{CA2\_BCD\_4B}}

\paragraph{Símbol}
\begin{center} \bsfsymbol{CA2_BCD_4B} \end{center}

\paragraph{Entrades i sortides}

\begin{where}
\item[\nodenamerange{x}{3}{0}] Entrada (Ca2)
\item[\nodenamerange{bcd}{3}{0}] Mòdul (binari natural)
\end{where}

\paragraph{Funció}

Extractor de mòdul de Ca2 de 4~bits.

Calcula el valor absolut de l'entrada i el retorna al
bus de sortida en binari natural (o en BCD, perquè la sortida és de 4~bits i
mai serà major que 9).

Nota: La sortida es diu $bcd$ per consistència amb el bloc \textsf{CA2\_BCD\_8B}.

\paragraph{Inespecificacions}

Cap.

\paragraph{Implementació}

\begin{figure}[b]
  \begin{center}
    \adjustbox{max width=\textwidth, max height=\textheight}{
      \bdfschematic{CA2_BCD_4B}
    }
  \end{center}
  \caption{\label{fig:\projectname-CA2_BCD_4B} Esquemàtic per al bloc \textsf{CA2\_BCD\_4B}}
\end{figure}

L'esquemàtic del bloc es pot veure a la figura~\ref{fig:\projectname-CA2_BCD_4B} (pàgina~\pageref{fig:\projectname-CA2_BCD_4B}).

Si $x_3 = 0$, l'entrada és positiva i per tant no cal fer cap operació, es
retorna directament.

Per contra, si $x_3 = 1$ llavors s'ha de convertir en positiva negant bit a bit
i sumant 1 al resultat. La primera part es pot fer amb una \textsf{XOR}, la segona
es pot fer entrant el resultat a un sumador, amb l'altra entrada nula i carry
d'entrada $x_3$. La sortida d'aquest sumador és el mòdul que busquem.

\vspace{1cm}
