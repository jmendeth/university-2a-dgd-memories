\chapter{Pràctica 1A: Multiplexor BCD}

\section{Especificació}

Es demana construir un multiplexor de 2 paraules de 4~bits BCD a la placa FPGA.
L'usuari introduïrà manualment les dues paraules mitjançant 4~switches per cada
una, i controlarà el bit de selecció mitjançant un altre switch. Dos displays
de set segments de la placa mostraran les dues entrades, i un tercer display,
la sortida del multiplexor (també passada a set segments).

Es farà servir una placa DE2 on es carregarà el resultat. Es farà servir el
Quartus 9.1 SP2 Web Edition per al desenvolupament.

\section{Implementació}

Només necessitem un multiplexor de 2~busos de 4~bits, i convertidors de BCD
a set segments. Les entrades ja estan en BCD, anirien directament a un multiplexor, d'aquí als convertidors i llavors als displays.

Per a construir el multiplexor de busos de 4~bits, farem servir 4~multiplexors d'un
bit, un per a cada un dels bits dels busos.

A continuació es descriuen els blocs que s'han emprat, començant pels menys
simples i acabant amb el bloc de més alt nivell jeràrquic, que es carrega
directament a la placa.

  \cclearpage
\section{Blocs lògica}
\inputblock{MUX_2_to_1}
  \cclearpage
\inputblock{MUX_2B4_to_1B4}

  \cclearpage
\section{Blocs de presentació}
\inputblock{BCD7seg}
  \cclearpage
\inputblock{P_Mux}

\section{Valoració general}

En aquest cas no ens vam trobar amb cap tipus de problema rellevant, més enllà de que era la primera vegada que treballavem amb el Quartus i ens haviem de familiaritzar amb el funcionament del programa.

