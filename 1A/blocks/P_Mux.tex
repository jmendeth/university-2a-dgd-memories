\subsection{\label{sub:\projectname-P_Mux} \textsf{P\_Mux}}

\paragraph{Entrades i sortides}

\begin{where}
\item[\nodenamerange{SW}{17}{14}] Switches on s'introdueix la segona xifra BCD
\item[\nodenamerange{SW}{13}{10}] Switches on s'introdueix la primera xifra BCD
\item[\nodenamesingle{SW}{0}] Switch on s'introdueix el bit de selecció
\item[\nodenamerange{HEX6}{6}{0}] Display set-segments 6
\item[\nodenamerange{HEX4}{6}{0}] Display set-segments 4
\item[\nodenamerange{HEX0}{6}{0}] Display set-segments 0
\end{where}

\paragraph{Funció}

Disseny a carregar a la placa FPGA.

Visualitza la primera entrada i la segona entrada (en BCD) als displays~4 i 6,
respectivament. Si $SW_0$ està desactivat, visualitza la primera entrada al
display~0; si està activat, visualitza la segona entrada al display~0.

\paragraph{Inespecificacions}


La sortida està inespecificada si $SW_{17..4}$ o bé $SW_{13..10}$ no tenen valors
vàlids en BCD.


\paragraph{Implementació}

\begin{figure}[b]
  \begin{center}
    \adjustbox{max width=\textwidth, max height=\textheight}{
      \bdfschematic{P_Mux}
    }
  \end{center}
  \caption{\label{fig:\projectname-P_Mux} Esquemàtic per al bloc \textsf{P\_Mux}}
\end{figure}

L'esquemàtic del bloc es pot veure a la figura~\ref{fig:\projectname-P_Mux} (pàgina~\pageref{fig:\projectname-P_Mux}).

La primera entrada es porta cap al primer conversor BCD a set-segments, i cap a l'entrada 0 d'un multiplexor 2:1. La segona entrada es porta a un segon conversor set-segments, i a l'entrada 1 del multiplexor.

Finalment, la sortida del multiplexor es porta a un tercer conversor BCD a set-segments. Les sortides dels tres conversors es porten als displays 4, 6 i 0 respectivament.

\vspace{1cm}
