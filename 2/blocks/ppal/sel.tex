\subsection{\label{sub:\projectname-sel} \textsf{sel}}

\paragraph{Símbol}

\begin{center} \bsfsymbol{sel} \end{center}

\paragraph{Entrades i sortides}

\begin{where}
\item[\nodenamerange{res}{7}{0}] Resultat actual (BCD)
\item[\nodenamebit{show}] Senyal que s'activa si s'ha de mostrar el resultat
\item[\nodenamerange{sel}{7}{0}] Sortida pels displays (BCD$^\dagger$)
\end{where}

\paragraph{Funció}

Controla la sortida del bloc, que posteriorment serà convertida a set-segments
i es mostrarà als displays del resultat.

Si $show$ és actiu, la sortida $sel$ és directament l'entrada $res$. En cas contrari,
la sortida són xifres BCD invàlides per tal d'apagar els displays que mostren
el resultat.

\paragraph{Inespecificacions}

Cap.

\paragraph{Notes}

El funcionament d'aquest bloc es basa en que els conversors a set-segments
apaguen els displays per a una xifra BCD invàlida. Això és el cas per a
\textsf{BCD7seg}.

\paragraph{Implementació}

\vhdlisting{sel}



% FIXME

\vspace{1cm}
