\subsection{\label{sub:\projectname-control} \textsf{control}}

\paragraph{Símbol}

\begin{center} \bsfsymbol{control} \end{center}

\paragraph{Entrades i sortides}

\begin{where}
\item[\nodenamebit{bcd}] Senyal que s'activa si s'ha premut una xifra decimal
\item[\nodenamebit{ast}] Senyal que s'activa si s'ha premut la tecla asterisc
\item[\nodenamebit{coi}] Senyal que s'activa si s'ha premut la tecla coixinet
\item[\nodenamebit{intro}] Senyal que s'activa si s'ha d'emmagatzemar una xifra nova
\item[\nodenamebit{show}] Senyal que s'activa si s'ha de mostrar el resultat
\item[\nodenamebit{clk}] Rellotge, flanc de pujada
\item[\nodenamebit{nrst}] Reset asíncron, actiu baix
\end{where}

\paragraph{Funció}

Bloc de control d'estat per al multiplicador.

El senyal $show$ indica si el multiplicador està en mode visualització o introducció,
i el senyal $intro$ s'activa quan el multiplicador està en mode introducció i l'usuari
prem una xifra decimal.

\paragraph{Inespecificacions}


El comportament del bloc no està definit si més d'una entrada està activa a la vegada.


\paragraph{Implementació}

\vhdlisting{control}



Es tracta d'una màquina de Mealy amb diagrama d'estats:

% TODO

Quan arriba el flanc de rellotge, es força l'estat \mintinline{vhdl}|st_show| si $ast$ és
actiu, o l'estat \mintinline{vhdl}|st_intro| si $coi$ és actiu.

La sortida $show$ indica l'estat actual de la màquina, i la sortida $intro$ és activa
només quan $bcd$ és activa i ens trobem en l'estat \mintinline{vhdl}|st_intro|.

\vspace{1cm}
