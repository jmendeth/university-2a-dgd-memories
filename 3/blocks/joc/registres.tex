\subsection{\label{sub:\projectname-registres} \textsf{registres}}

\paragraph{Símbol}

\begin{center} \bsfsymbol{registres} \end{center}

\paragraph{Entrades i sortides}

\begin{where}
\item[\nodenamebit{eshft}] Habilitació d'emmagatzemament
\item[\nodenamerange{keycode}{3}{0}] Xifra que s'està introduint (BCD)
\item[\nodenamerange{num}{7}{0}] Nombre enmagatzemat (BCD, dos xifres)
\item[\nodenamebit{clk}] Rellotge, flanc de pujada
\item[\nodenamebit{nrst}] Reset asíncron, actiu baix
\end{where}

\paragraph{Funció}

\emph{Shift register} per a dues xifres BCD.

Emmagatzema un nombre BCD de dues xifres, memoritzant les dues últimes
xifres carregades. Quan s'habilita la càrrega ($eshft = 1$), la xifra
BCD present a $keycode$ esdevé la xifra de menys pes del nombre
emmagatzemat; la xifra anterior esdevé la de més pes, i aquesta es descarta.

\paragraph{Inespecificacions}


La sortida no pertanyirà al seu codi si es carrega una xifra no BCD.


\paragraph{Implementació}

\begin{contendfig}
  \begin{center}
    \adjustbox{max width=\textwidth, max height=\textheight}{
      \bdfschematic{registres}
    }
  \end{center}
  \caption{\label{fig:sch-\projectname-registres} Esquemàtic per al bloc \textsf{registres}}
\end{contendfig}

L'esquemàtic del bloc es pot veure a la figura~\ref{fig:sch-\projectname-registres} (pàgina~\pageref{fig:sch-\projectname-registres}).

Dos biestables D amb habilitació de càrrega encadenats, de forma similar a un \emph{shift
register} però on les sortides es concatenen per formar la sortida $num$.

\vspace{1cm}
