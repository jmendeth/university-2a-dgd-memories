\subsection{\label{sub:\projectname-control} \textsf{control}}

\paragraph{Símbol}

\begin{center} \bsfsymbol{control} \end{center}

\paragraph{Entrades i sortides}

\begin{where}
\item[\nodenamebit{bcd}] Senyal que s'activa si s'ha premut una xifra decimal
\item[\nodenamebit{ast}] Senyal que s'activa si s'ha premut la tecla asterisc
\item[\nodenamebit{coi}] Senyal que s'activa si s'ha premut la tecla coixinet
\item[\nodenamebit{ngtx}] Indica que el nombre introduït és major que la solució
\item[\nodenamebit{neqx}] Indica que el nombre introduït coincideix amb la solució
\item[\nodenamebit{nltx}] Indica que el nombre introduït és menor que la solució
\item[\nodenamebit{eshft}] Senyal que s'activa quan s'ha d'enregistrar la xifra premuda
\item[\nodenamebit{ecnt}] Senyal que d'habilitació del comptador
\item[\nodenamerange{comp}{2}{0}] Sortida per als indicadors de l'estat del joc
\item[\nodenamebit{clk}] Rellotge, flanc de pujada
\item[\nodenamebit{nrst}] Reset asíncron, actiu baix
\end{where}

\paragraph{Funció}

Màquina de control del joc.

Emmagatzema i actualitza l'estat del joc, habilitant el comptador o el registre segons convingui.
La sortida $comp$ son tres bits (aproximadament, «major» / «igual» / «menor») que tenen
combinacions amb diversos significats i que controlen els LEDs de la placa:

\begin{description}
\item[111] El joc s'acaba d'inicialitzar i es troba en espera (no s'ha jugat cap partida).
\item[000] S'ha començat una nova partida, s'està esperant que es comprovi un nombre.
\item[100] L'usuari ha comprovat un nombre que ha resultat ser major que la solució.
\item[001] L'usuari ha comprovat un nombre que ha resultat ser menor que la solució.
\item[010] L'usuari ha comprovat un nombre que ha resultat ser igual a la solució;
s'ha acabat la partida i el joc es troba en espera.
\end{description}

\paragraph{Inespecificacions}


El comportament del bloc no està definit si més d'una de les entrades $bcd, ast, coi$ està activa a la vegada.
El comportament del bloc no està definit si cap de les entrades $ngtx, neqx, nltx$ està activa, o més d'una ho està.


\paragraph{Implementació}

\vhdlisting{control}



Hem pres una forma poc usual de dissenyar aquest bloc: en comptes d'assignar
$comp$ de forma combinacional fora del \mintinline{vhdl}|process|, es fa dins
i només en certs casos (\emph{latch}). Hem vist que això simplifica molt el
codi i ens ha permés fer el disseny en només \textbf{2 estats} (repós, jugant)
en comptes de 5.

El diagrama de transició d'estats (només depen de les entrades $coi$, $neqx$; sortides $eshft$, $ecnt$) és el següent. Es tracta
d'una màquina de Mealy, però es representa la sortida dins de l'estat per simplicitat:

\begin{center} \begin{tikzpicture}[shorten >=1pt, node distance=2cm, >=stealth', auto]
  \node[state with output, initial above, initial text=reset] (st_idle) at (0,0) {idle \nodepart{lower} $0,1$};
  \node[state with output]          (st_playing) at (4,0) {playing \nodepart{lower} $bcd,0$};

  \path[->]
    (st_idle) edge [loop left, align=center] node {$0,-$} (st_idle)
    (st_idle) edge [bend left, align=center] node {$1,-$} (st_playing)
    
    (st_playing) edge [loop right, align=center] node {$0,0$} (st_playing)
    (st_playing) edge [bend left, align=center] node {$1,-$ \\ $-,1$} (st_idle)
  ;
\end{tikzpicture} \end{center}

Respecte a $comp$, se li assigna un valor només en els següents casos:

\begin{itemize}
\item En reset: \mintinline{vhdl}|"111"|
\item En estat \emph{idle}, $coi$ s'activa: \mintinline{vhdl}|"000"|
\item En estat \emph{playing}, $bcd$ s'activa: \mintinline{vhdl}|"000"|
\item En estat \emph{playing}, $ast$ s'activa: \mintinline{vhdl}|ngtx & neqx & nltx|
\item En estat \emph{playing}, $coi$ s'activa: \mintinline{vhdl}|"111"|
\end{itemize}

\paragraph{Simulació}

\begin{center}
  \begin{tikztimingtable}[timing/rowdist=4ex]
  $\nodenamebit{clk}$  &  [] 2.25L 2.25C 2.25C 2.25C 2.25C 2.25C 2.25C 2.25C 2.25C 2.25C 2.25C 2.25C 2.25C 2.25C 2.25C 2.25C 2.25C 2.25C 2.25C 2.25C \\
  $\nodenamebit{nrst}$  &  [] 2.25L 42.75H \\
  $\nodenamebit{coi}$  &  [] 9L 4.5H 27L 4.5H \\
  $\nodenamebit{bcd}$  &  [] 4.5L 4.5H 4.5L 9H 4.5L 4.5H 13.5L \\
  $\nodenamebit{ast}$  &  [] 22.5L 4.5H 4.5L 4.5H 9L \\
  $\nodenamebit{ngtx}$  &  [] 18L 9H 18L \\
  $\nodenamebit{neqx}$  &  [] 31.5L 13.5H \\
  $\nodenamebit{nltx}$  &  [] 13.5L 4.5H 27L \\
  $\nodenamebit{comp}$  &  [] 11.250022D{111} 0D{} 0.000005D{} 13.499995D{000} 4.5D{001} 4.5D{000} 9D{010} 2.249978D{000} \\
  $\nodenamebit{ecnt}$  &  [] 11.250022H 22.5L 9H 2.249978L \\
  $\nodenamebit{eshft}$  &  [] 13.500023L 9H 4.5L 4.5H 13.499977L \\
\extracode
  \begin{pgfonlayer}{background}
    \vertlines[help lines]{2.25,6.75,11.25,15.75,20.25,24.75,29.25,33.75,38.25,42.75}
  \end{pgfonlayer}
\end{tikztimingtable}

\end{center}

En aquesta simulació comprovem que les sortides es comporten correctament. Ens basem en una situació real que es donaria en el joc (p.e. no posariem les sortides $ngtx$, $netx$ o $nltx$ alhora).

\vspace{1cm}
