\subsection{\label{sub:\projectname-comparador_single} \textsf{comparador\_single}}

\paragraph{Símbol}

\begin{center} \bsfsymbol{comparador_single} \end{center}

\paragraph{Entrades i sortides}

\begin{where}
\item[\nodenamerange{numx}{3}{0}] Primer nombre
\item[\nodenamerange{num}{3}{0}] Segon nombre
\item[\nodenamebit{ngtxi}] Entrada «major que» encadenada
\item[\nodenamebit{neqxi}] Entrada «igual que» encadenada
\item[\nodenamebit{nltxi}] Entrada «menor que» encadenada
\item[\nodenamebit{ngtx}] Indica que $num$ és major que $numx$
\item[\nodenamebit{neqx}] Indica que $num$ és igual que $numx$
\item[\nodenamebit{nltx}] Indica que $num$ és menor que $numx$
\end{where}

\paragraph{Funció}

Comparador de busos de 4~bits encadenable.

Si $neqxi = 1$, compara el valor en binari natural de les dues entrades i activa la sortida corresponent:
%
\begin{align*}
  ngtx &= \begin{cases}
    1 & \text{si $num > numx$} \\
    0 & \text{altrament}
  \end{cases} \\
  neqx &= \begin{cases}
    1 & \text{si $num = numx$} \\
    0 & \text{altrament}
  \end{cases} \\
  nltx &= \begin{cases}
    1 & \text{si $num < numx$} \\
    0 & \text{altrament}
  \end{cases}
\end{align*}

En cas contrari, les entrades es porten directament a les sortides:
%
\begin{align*}
  ngtx &= ngtxi \\
  neqx &= neqxi \\
  nltx &= nltxi
\end{align*}

\paragraph{Inespecificacions}

Cap.

\paragraph{Implementació}

\vhdlisting{comparador_single}



Es fa servir el paquet \mintinline{vhdl}|std_logic_unsigned| i s'assigna cada sortida a la comparació que toca.

Però en cas que $neqxi$ no estigui activat, les sortides reben el valor directament de les entrades, i no de les comparacions.

\paragraph{Simulació}

\begin{center}
  \begin{tikztimingtable}[timing/rowdist=4ex]
  $\nodenamebit{ngtxi}$  &  [] 8.1H 37.8L \\
  $\nodenamebit{neqxi}$  &  [] 8.1L 37.8H \\
  $\nodenamebit{nltxi}$  &  [] 45.9L \\
  $\nodenamebit{numx}$  &  [] 0D{} 2.7D{7} 2.7D{8} 2.7D{9} 2.7D{0} 2.7D{1} 2.7D{2} 2.7D{3} 2.7D{4} 2.7D{5} 2.7D{6} 2.7D{7} 2.7D{8} 2.7D{9} 2.7D{0} 2.7D{1} 2.7D{2} 2.7D{3} \\
  $\nodenamebit{num}$  &  [] 8.1D{9} 27D{0} 10.8D{1} \\
  $\nodenamebit{ngtx}$  &  [] 8.544116H 26.976071L 2.690078H 7.689735L \\
  $\nodenamebit{neqx}$  &  [] 8.42022L 0.115358H 0.037901L 2.672325H 24.3L 0.001181H 2.698819L 2.689875H 4.964321L \\
  $\nodenamebit{nltx}$  &  [] 11.152181L 8.1H 0.010125L 10.8H 0.001856L 5.4H 5.377961L 5.057876H \\
\extracode
\end{tikztimingtable}


\begin{tikztimingtable}[timing/rowdist=4ex]
  $\nodenamebit{ngtxi}$  &  [] 45.9L \\
  $\nodenamebit{neqxi}$  &  [] 45.9H \\
  $\nodenamebit{nltxi}$  &  [] 45.9L \\
  $\nodenamebit{numx}$  &  [] 2.7D{4} 2.7D{5} 2.7D{6} 2.7D{7} 2.7D{8} 2.7D{9} 2.7D{0} 2.7D{1} 2.7D{2} 2.7D{3} 2.7D{4} 2.7D{5} 2.7D{6} 2.7D{7} 2.7D{8} 2.7D{9} 2.7D{0} 0D{} \\
  $\nodenamebit{num}$  &  [] 16.2D{4} 27D{5} 2.7D{6} \\
  $\nodenamebit{ngtx}$  &  [] 0.399701H 10.8L 0.015458H 5.4L 13.495106H 8.089436L 0.015458H 5.4L 2.284841H \\
  $\nodenamebit{neqx}$  &  [] 0.445804L 2.7H 24.289875L 0.010125H 2.7L 2.689875H 13.064321L \\
  $\nodenamebit{nltx}$  &  [] 3.052181L 8.1H 0.011981L 5.4H 16.177961L 5.4H 0.022039L 5.4H 2.335838L \\
\extracode
\end{tikztimingtable}

\end{center}

En aquest cas hem fet la simulació de totes les comparacions possibles, en funció de les entrades $ngtxi$, $neqxi$ i $nltxi$. Hem centrat la simulació en una de les parts clau del comparador, quan $neqxi = 1$. Hem de tenir en compte com interpretar la sortida. En el nostre cas comparem $num$ respecte a $numx$.

\vspace{1cm}
