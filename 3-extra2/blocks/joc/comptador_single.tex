\subsection{\label{sub:\projectname-comptador_single} \textsf{comptador\_single}}

\paragraph{Símbol}

\begin{center} \bsfsymbol{comptador_single} \end{center}

\paragraph{Entrades i sortides}

\begin{where}
\item[\nodenamebit{ecnt}] Habilitació del compte
\item[\nodenamerange{numx}{3}{0}] Nombre de sortida (BCD)
\item[\nodenamebit{tc}] Sortida \emph{tail count}
\item[\nodenamebit{clk}] Rellotge, flanc de pujada
\item[\nodenamebit{nrst}] Reset asíncron, actiu baix
\end{where}

\paragraph{Funció}

Comptador BCD d'una xifra (ascendent cíclic).

El valor BCD de la sortida, inicialitzada a zero, s'incrementa en cada cicle de rellotge que sempre que $ecnt = 1$.
Quan la sortida arriba al valor 9, en el següent cicle s'estableix a 0 i es segueix incrementant de la mateixa forma.

La sortida $tc$ s'usa per encadenar comptadors.

\paragraph{Inespecificacions}

Cap.

\paragraph{Implementació}

\vhdlisting{comptador_single}



Com en el comptador anterior, es fa servir el paquet \mintinline{vhdl}|numeric_std| i es defineix una senyal intermèdia
$n$ de tipus \mintinline{vhdl}|unsigned| de 4~bits. S'inicialitza en 0, i en arribar el flanc de rellotge i si $ecnt = 1$,
s'incrementa (si el seu valor és diferent de 9) o es reinicialitza a zero (cas contrari).

La senyal intermèdia es converteix a vector lògic i s'assigna a la sortida $numx$.
S'activa $tc$ si $ecnt = 1$ i la senyal intermèdia té el valor 9.

\paragraph{Simulació}

\begin{center}
  \begin{tikztimingtable}[timing/rowdist=4ex]
  $\nodenamebit{clk}$  &  [] 1.575L 1.575C 1.575C 1.575C 1.575C 1.575C 1.575C 1.575C 1.575C 1.575C 1.575C 1.575C 1.575C 1.575C 1.575C 1.575C 1.575C 1.575C 1.575C 1.575C 1.575C 1.575C 1.575C 1.575C 1.575C 1.575C 1.575C 1.575C \\
  $\nodenamebit{nrst}$  &  [] 44.1H \\
  $\nodenamebit{ecnt}$  &  [] 1.1025L 42.9975H \\
  $\nodenamebit{numx}$  &  [] 1.589721D{0} 3.15D{1} 0.000305D{} 3.149695D{2} 3.149993D{3} 0.000007D{} 0.000305D{} 3.149695D{4} 3.15D{5} 0.000305D{} 3.149695D{6} 3.149993D{7} 0.000007D{} 0.000305D{} 0.000365D{} 3.149329D{8} 3.15D{9} 0.000671D{} 3.149329D{0} 3.15D{1} 0.000305D{} 3.149695D{2} 3.149993D{3} 0.000007D{} 0.000305D{} 1.559974D{4} \\
  $\nodenamesingle{numx}{3}$  &  [] 23.640392L 6.3H 14.159608L \\
  $\nodenamesingle{numx}{2}$  &  [] 11.039714L 12.6H 18.9L 1.560286H \\
  $\nodenamesingle{numx}{1}$  &  [] 4.740026L 6.3H 6.3L 6.3H 12.6L 6.3H 1.559974L \\
  $\nodenamesingle{numx}{0}$  &  [] 1.589721L 3.15H 3.15L 3.15H 3.15L 3.15H 3.15L 3.15H 3.15L 3.15H 3.15L 3.15H 3.15L 3.15H 1.560279L \\
  $\nodenamebit{tc}$  &  [] 26.793785L 3.15H 14.156215L \\
\extracode
  \begin{pgfonlayer}{background}
    \vertlines[help lines]{1.575,4.725,7.875,11.025,14.175,17.325,20.475,23.625,26.775,29.925,33.075,36.225,39.375,42.525}
  \end{pgfonlayer}
\end{tikztimingtable}

\end{center}

Ens centrem en el canvi més rellevant del bloc: el pas de 9 a 0. Hem d'assegurar-nos que $tc$ s'activi quan pertoqui.

\vspace{1cm}
