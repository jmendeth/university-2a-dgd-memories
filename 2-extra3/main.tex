\chapter{Extra 3: Calculadora sel·leccionable}

\section{Especificació}

Es demana ampliar el disseny del multiplicador de l'apartat 2 en una calculadora genèrica que sigui capaç de realitzar fins a quatre operacions, entre les quals figura la multiplicació de A per B, com es va dissenyar originalment. Per a passar al mode d'introducció, l'usuari pot premer qualsevol de les quatre tecles següents, i això determina l'operació que es mostrarà als displays de sortida quan es premi \texttt{\#}:

\begin{description}
\item[*] Producte de $A$ per $B$
\item[B] Quadrat de $A$
\item[C] Quadrat de $B$
\item[D] Suma de $A$ amb $B$
\end{description}

Quatre LEDs addicionals de la placa mostraran en tot moment el nombre d'operació sel·leccionada.

Exemple: si es prem \texttt{D}, \texttt{2}, \texttt{3}, \texttt{A}, \texttt{\#} els displays mostraran:

\begin{center}
\sevenseg{-}\sevenseg{3}\sevenseg{ }\sevenseg{2} \sevenseg{ }\sevenseg{-}\sevenseg{0}\sevenseg{1}
\end{center}

\section{Implementació}

Per a aquesta millora hem d'afegir les noves ordres (\texttt{B} -- primera xifra al quadrat, \texttt{C} -- segona xifra al quadrat, \texttt{D} --  suma de les xifres) a \textsf{keygroup}. Com aquest cas tindrem 4 operacions diferents amb les que tractar, generem una nova sortida de dos bits (ja que tindrem 4 operacions disponibles) a \textsf{keygroup} que anomenarem $selop$, que indicarà la operació a realitzar. Aquesta sortida anirà conectada a un registre que conectarà amb un nou bloc que farà les operacions que li indiquem, i al bloc \textsf{leds}. A més, utilitzarem la sortida $ast$ per indicar que es realitza una operació qualsevol (però no li canviem el nom per no liar-nos massa).

El nou bloc a crear s'anomena \textsf{Calculadora}, i realitzarà totes les operacions que li indiquem, en funció del que l'indiqui $selop$. \textsf{Calculadora} la formaran un bloc que s'encarregarà de fer les multiplicacions i un altre de la suma. Reutilitzarem \textsf{AperB}, l'únic que hem de fer es afegir uns multiplexors a cadascuna de les entrades. Per a la suma hem de generar un nou bloc, que anomenarem \textsf{AmesB}, que ens donarà coma sortida la suma de les entrades, signe inclòs. El bloc \textsf{Calculadora} substituirà el bloc \textsf{AperB} a \textsf{ppal}.

També haurem de modificar LEDs perquè sigui capaç de indicar la operació que es realitza. Per acomplir aquesta fi, li hem afegir selop a l'entrada i li hem donat noves sortides cap a altre LEDs de la placa.

Finalment, ara el rang de sortida, degut a la suma serà $\left[-18,81\right]$, per tant hem d'afegir aquests nous números al conversor Ca2 BCD 8 bits.

\section{Blocs combinacionals}
\inputblock{ppal/CA2_BCD_8B}
\inputblock{ppal/AmesB}
\inputblock{ppal/Calculadora}

\section{Blocs seqüencials}
\inputblock{ppal/keygroup}
\inputblock{ppal}

\section{Blocs d'adaptació a la placa}
\inputblock{leds}
\inputblock{calc}

\section{Valoració general}

El que ens va donar més maldecaps a l'hora de descriure \textsf{Calculadora} va ser el bloc \textsf{AmesB}, ja que resultava molt tediós passar de format mòdul -- signe a Ca2, fent les extensions de signe necessàries... Després de diversos intents, vam acabar utilitzant \mintinline{vhdl}|ieee.numeric_std| que va simplificar força el codi. Però continuaven obtenint un resultat incorrecte ja que ens haviem oblidat d'afegir les entrades corresponents a la TdV de \textsf{CA2\_BCD\_8B}. Una vegada afegits, la calculadora va començar a donar els resultats que voliem.
